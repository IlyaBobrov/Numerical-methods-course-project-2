\documentclass[14pt, titlepage, a4paper]{extarticle} %comment
\usepackage{fancyvrb}
\usepackage[utf8]{inputenc}
\usepackage[T2A]{fontenc}
\usepackage[russian]{babel}
\usepackage{amssymb, amsfonts}
\usepackage{amsmath}

\usepackage{graphicx}

\usepackage{lipsum}
\usepackage{style}
\textwidth 16cm
\textheight 25cm
\oddsidemargin -2.5mm
\evensidemargin -3mm
\topmargin -20mm
\parindent 1.25cm

\usepackage{xcolor}
\usepackage{hyperref}
% Цвета для гиперссылок
\definecolor{linkcolor}{HTML}{000000} % цвет ссылок
\definecolor{urlcolor}{HTML}{000000} % цвет гиперссылок
\hypersetup{pdfstartview=FitH,  linkcolor=linkcolor,urlcolor=urlcolor, colorlinks=true}


\usepackage{comment}
%\usepackage{autonum}
\usepackage{amsthm}
\newtheorem{theorem}{Теорема}
\newtheorem{definition}{Определение}
\newtheorem{corollary}{Следствие}
\newtheorem{lemma}{Лемма}


\begin{document}
	
	%%%%%%%%%%%%%%%%%%%%PAGE%1%%%%%%%%%%%%%%%%%%%%
	%%%%%%%%%%%%%%%%%%%Вкладка%%%%%%%%%%%%%%%%%%%%
	\begin{comment}
	\fefutitlepage{Б8118-01.03.02миопд}{Бобров И.И.}{89376650756}{15}{\hspace{15pt}мая\hspace{15pt}}
	\defaultfont	
	\end{comment}

	%%%%%%%%%%%%%%%%%%%%PAGE%3%%%%%%%%%%%%%%%%%%%%
	%%%%%%%%%%%%%%%%%%Содержание%%%%%%%%%%%%%%%%%%

	\tableofcontents
	\pagebreak
	
	%%%%%%%%%%%%%%%%%%%%PAGE%4%%%%%%%%%%%%%%%%%%%%
	%%%%%%%%%%%%%%%%%%%Введение%%%%%%%%%%%%%%%%%%%
	
	\section*{Введение}
	\addcontentsline{toc}{section}{Введение}
	
	Объектом исследования является точный метод решения СЛАУ - построение QR-разложения методом отражений.\\
	\textit{Цель работы} – ознакомиться с алгоритмами метода отражений и построения QR-разложения, решить типовые задачи, сформулировать выводы по полученным решениям, отметить достоинства и недостатки метода, приобрести практические навыки и компетенции, а также опыт самостоятельной профессиональной деятельности, а именно:
	\begin{itemize}
		\item создать алгоритм решения поставленной задачи и реализовать его, протестировать;
		\item освоить теорию вычислительного эксперимента; современных компьютерных технологий; 
		\item приобрести навыки представления итогов проделанной работы в виде отчета, оформленного в соответствии с имеющимися требованиями, с привлечением современных средств редактирования и печати.	
	\end{itemize}
	Работа над курсовым проектом предполагает выполнение следующих задач:
	\begin{itemize}
		\item дальнейшее углубление теоретических знаний обучающихся;
		\item получение и развитие прикладных умений и практических навыков по направлению подготовки;
		\item овладение методикой решения конкретных задач;
		\item развитие навыков самостоятельной работы;
		\item развитие навыков обработки полученных результатов, анализа и осмысления их с учетом имеющихся литературных данных;
		\item приобретение навыков оформления описаний программного продукта;
		\item повышение общей и профессиональной эрудиции.
	\end{itemize}

	
	\pagebreak
	
	%%%%%%%%%%%%%%%%%%%%PAGE%5%%%%%%%%%%%%%%%%%%%%
	%%%%%%%%%%%%%%%%Основная%часть%%%%%%%%%%%%%%%%
	%%%%%%%%%%%%%%%Постановка%задачи%%%%%%%%%%%%%%%
	
	\section*{Основная часть}
	\addcontentsline{toc}{section}{Основная часть}
	
	\subsection*{Метод отражений}
	\addcontentsline{toc}{subsection}{Метод отражений.}
	Метод отражения представляет собой алгоритм подбора унарных матриц преобразований \textit{Р}, таких что в результате всех этих преобразований исходная матрица \textit{А} приводится к треугольному виду. Система с треугольной матрицей в дальнейшем решается, например, методом \textit{Гаусса}. Не смотря на трудоемкость метода, он имеет широкое распространение благодаря своей устойчивости к накоплению вычислительной погрешности.\\

	В n - мерном евклидовом пространстве рассмотрим гиперплоскость $(p,x) = p_1 x_1 + p_2 x_2 + ... + p_n x_n = 0$, проходящую через начало координат ортогонально заданному вектору нормали $p = (p_1 ,p_2 ,...,p_n)^*$. Поставив в соответствие каждому элементы x рассматриваемого пространства элемент
	\begin{equation}\label{eq:1}
		y = x-2 \frac{(p,x)}{(p,p)}p,
	\end{equation}
	мы определим некоторое преобразование n пространства, которое называется преобразованием ортогонального отражения относительно гиперплоскости $(p,x)=0$
	\begin{definition}\label{def:1}
		Матрицей отражения называется матрица вида $F = I - \frac{2}{(p,p)}pp^*$, относительно гиперплоскости с нормалью p
		Всякая матрица отражения целиком определяется соответствующим вектором нормали
	\end{definition}
	~\\
	Рассмотрим некоторые свойства матрицы
	
	\begin{enumerate}
		\item $F^2 = I$
		\item $F^* = F$
		\item Матрица \textit{F} - ортогональна.
		\item Матрица отражения не изменяется, если в место нормали \textit{p}, определяющего эту матрицу, использовать любой коллинеарный вектор $\beta p$ $(\beta \neq 0)$.
		\item Если $y=Fx$ и \textit{F} - матрица отражения, то в качестве определяющего ее вектора нормали можно взять разность исходного и отраженного векторов:
		$$p = x - y$$
		\item Если первые \textit{k} компонент вектора нормали нулевые, то первые \textit{k} компонент отраженного вектора совпадают с соответствующими компонентами исходного вектора.
		\item Если $p_1 = p_2 = ... = p_k = 0$ и $x_{k+1} = x_{k+2} =...= x_n = 0$, то и $y_{k+1} = y_{k+2} =...=y _n = 0$ 
	\end{enumerate}
	
	
	
	%%%%%%%%%%%%%%%%%%%%%%%%%%%%%%%%%%%%%%%%%%%%%%%%%%%%%%%%%%%%%%%%%%%%%%%%%%%%%%%%%%%%%%%
	~\\
	\subsection*{Построение QR-разложения методом отражений}
	\addcontentsline{toc}{subsection}{Построение QR-разложения методом отражений}
	
	\begin{theorem}\label{th:1}(О QR-разложении)
		Всякая невырожденная матрица $A\in M_n$ может быть представлена в виде $A = QR$, где Q - унитарная, а матрица R - верхняя треугольная с вещественным положительными элементами на главной диагонали. Это разложение единственно.
	\end{theorem}
	~\\
	Коротко по шагам опишем алгоритм приведения матрицы \textit{A} к верхней треугольной форме с помощью преобразований отражения.
	
	\begin{enumerate}
		\item Строим матрицу $A_1$ по формуле (\ref{eq:1}). Для этого определим матрицу отражения $F_1$ так, чтобы первый столбец матрицы 
		\begin{equation}\label{eq:2}
			A_1 = F_1 A
		\end{equation}
		имел вид $(a_{11}^{(1)},0,0,...,0)^*$. Для определения элемента $a_{11}^{(1)}$ воспользуемся свойством сохранения длины вектора при ортогональном преобразовании. Так мы построим первый столбец матрицы $A_1$. Для определения остальных необходимо воспользоваться формулой $a_j^{(1)} = F_1a_j, j=2,3,...,n$, где по определению матрицы $F_1$:
		$$a_j^{(1)} = a_j - 2\frac{(p^{(1)},p_j)}{(p^{(1)},p^{(1)})}p^{(1)}, j=2,3,...,n$$
		
		\item Пусть в результате выполнения $k-1$ шагов мы получили матрицу $A_k$. На \textit{k}-м шаге определим матрицу отражения $F_k$ так, чтобы \textit{k}-й столбец матрицы
		\begin{equation}\label{eq:3}
			A_k = F_k A_{k-1}
		\end{equation}
		имел вид $(a_{1k}^{(1)},a_{2k}^{(2)},...,a_{k-1k}^{(k-1)},a_{kk}^{(k)}, 0,...,0)^*$.Согласно свойству 5 $P^{(k)} = a_k^{(k-1)}-a_k^{(k)}$, т.е. 
		$$P_l^{(k)} = 0,~~~ l=1,2,..,k-1,~~~ P_k^{(k)} = a_{kk}^{(k-1)} - a_{kk}^{(k)},$$
		$$P_l^{(k)} = a_{lk}^{(k-1)},~~~ l=k-1,...,n$$
		Элементы $a_{kk}^{(k)}$ определены из условия равенства длин столбцов $a_k^{(k-1)}$
		\begin{equation*}
		a{kk}^{(k)} = -\sigma_k\sqrt{\sum_{l=k}^{n}[ a_{lk}^{(k-1)} ]^2}, 
		\end{equation*}
		где 
		
		\begin{equation*}
		\sigma = 
		\begin{cases}
		1 &\text{, если $a_{kk}^{(k-1)}\geq 0$}\\
		-1 &\text{, если $a_{kk}^{(k-1)} < 0$}
		\end{cases}
		\end{equation*}
		Тогда 
		\begin{equation*}
			P_k^{(k)}=a_{kk}^{(k-1)}+\sigma_k\sqrt{\sum_{l=k}^{n}[ a_{lk}^{(k-1)} ]^2}
		\end{equation*}
		Полностью определив вектор нормали $P^{(k)}$, а значит и матрицу отражения $F_k$, можем приступить к выполнению \textit{k}-го шага, состоящего в вычислении матрицы $A_k$ по формуле (\ref{eq:3})
		
		\item Определив \textit{k}-тый столбец матрицы, определяем остальные воспользовавшись формулами $a_j^{(k)} = F_k a_j^{(k-1)}$. По определению матрицы отражения $F_k$ получаем
		\begin{equation}\label{eq:4}
			a_{ij}^{(k)} = a_{ij}^{(k-1)} - 2 \frac{p_i^{(k)}}{\sum_{l=k}^{n}(p_l^{(k)})^2}\sum_{l=k}^{n}p_l^{(k)}a_{lj}^{(k-1)}.
		\end{equation}
		
		\item В результате выполнения \textit{n}-1 шагов мы придем к матрице $A-{n-1}$, имеющий требуемую верхнюю треугольную матрицу которую будем обозначать за \textit{R}. Последовательно использование рекуррентной формулы (\ref{eq:3}) дает:
		$$R = A_{n-1} = F_{n-1}A_{n-2} = F_{n-1}F_{n-2}A_{n-3} = ... = F_{n-1}F_{n-2} \cdot \cdot \cdot F_2F_1A$$
		Обозначив через \textit{Q} произведение матриц и вычислив $Q^*$ получим $R = Q^*A$ и 
		$$A = Q\cdot R$$
		
	\end{enumerate}
	
	\pagebreak
	\subsection*{Реализация метода на языке MATLAB}
	\addcontentsline{toc}{subsection}{Реализация метода на языке MATLAB}
	
	Метод QR-разложения для удобства использования был вынесен в отдельную функцию:
	\begin{Verbatim}[numbers=left,xleftmargin=0mm]
function [Q,R] = functionQR(A)
    [n, m] = size(A);
    %Проверка корректности матрицы
    if (n ~= m)
    error('Error functionQR: Матрица не является квадратной');
    terminateExecution
    pause(0.1)
    end
    %Инициализация матриц
    I = eye(n,m);
    Q = I;
    R = A;
    %Главный цикл
    for k = 1:n-1
        x = A(:,k); %выбираем новый столбец на каждом шаге
        if (k > 1) 
            for i = 1:k-1
                x(i) = 0;
            end
        end
        %вычисляем норму
        norm_x = norm(x(:));
    
        a = zeros(n,1);
        a(k,1) = 1;
    
        p = x(:) - norm_x * a;
    
        norm_p = norm(p(:));
        %p' - транспонированная матрица p
        F = I - (2*p.*p')/(norm_p^2);
        Q = Q*F;
        A = F*A;
        R = A;
    end

end
	\end{Verbatim}

	~\\
	Main файл:	
	\begin{Verbatim}[numbers=left,xleftmargin=0mm]
clear all %осчистить буфер
clc %очистить консоль
format Long

%...
%...инициализация матрицы А...
%...

try
    [Q,R] = functionQR(A)
    A	
    Q*R %проверка
catch e
    fprintf('(%s)\n',e.message);
end
	\end{Verbatim}
	\pagebreak
	\subsection*{Вычислительный эксперимент}
	\addcontentsline{toc}{subsection}{Вычислительный эксперимент}
	
	Проведем вычислительные эксперименты, с помощью QR-разложения рассмотрим матрицы разных размеров и проанализируем вычислительную погрешность метода.
	 
	\subsubsection*{Эксперимент 1}
	\addcontentsline{toc}{subsubsection}{Эксперимент 1}

	Матрица A размера 3х3:
	\begin{Verbatim}[numbers=left,xleftmargin=0mm]
A = [   1     2     4
	3     3     2
	4     9     3]
	\end{Verbatim}
	Результат вычисления:
	\begin{Verbatim}[numbers=left,xleftmargin=0mm]
Q =

0.196116135138184   0.063966030264690   0.978492109580163
0.588348405414552  -0.805971981335094  -0.065232807305345
0.784464540552736   0.588487478435148  -0.195698421916032


R =

5.099019513592784   9.217458351494649   4.314554973040049
-0.000000000000000   3.006403422440430   0.409382593694015
-0.000000000000000   0.000000000000000   3.196407557961867


A =

1     2     4
3     3     2
4     9     3


Q*R =

1.000000000000000   2.000000000000000   4.000000000000000
3.000000000000000   3.000000000000002   2.000000000000000
4.000000000000000   8.999999999999998   3.000000000000000
	\end{Verbatim}
	
	\subsubsection*{Эксперимент 2}
	\addcontentsline{toc}{subsubsection}{Эксперимент 2}
	
	Матрица A размера 3х3:
	\begin{Verbatim}[numbers=left,xleftmargin=0mm]
A = [   1.4  1    1
	1    0.9  1
	1    1    1.4]
	\end{Verbatim}
	~\\
	Результат вычисления:
	\begin{Verbatim}[numbers=left,xleftmargin=0mm]
Q =

0.703526470681448  -0.680413817439772   0.205152484965555
0.502518907629606   0.272165526975908  -0.820609939862218
0.502518907629606   0.680413817439772   0.533396460910442


R =

1.989974874213240   1.658312395177700   1.909571848992503
-0.000000000000000   0.244948974278318   0.544331053951817
0.000000000000000  -0.000000000000000   0.131297590377955


A =

1.400000000000000   1.000000000000000   1.000000000000000
1.000000000000000   0.900000000000000   1.000000000000000
1.000000000000000   1.000000000000000   1.400000000000000


Q*R =

1.400000000000000   1.000000000000000   1.000000000000000
1.000000000000000   0.900000000000000   1.000000000000000
1.000000000000000   1.000000000000000   1.400000000000000
	\end{Verbatim}

\pagebreak
	\subsubsection*{Эксперимент 3}
	\addcontentsline{toc}{subsubsection}{Эксперимент 3}
	
	Матрица A размера 5х5:
	\begin{Verbatim}[numbers=left,xleftmargin=0mm]
A = [    1  3 -2  0 -2
	 3  4 -5  1 -3
	-2 -5  3 -2  2
	 0  1 -2  5  3
	-2 -3  2  3  4];
	\end{Verbatim}
	~\\
	Результат вычисления:
	\fontsize{9pt}{12pt}\selectfont
	\begin{Verbatim}[numbers=left,xleftmargin=0mm]
Q =
	
 0.235702260395516   0.496956188056659  -0.007875114405325  -0.759703701182689   0.346795704383812
 0.707106781186548  -0.453742606486515  -0.441006406698203   0.067860243460763   0.308262848341167
-0.471404520791032  -0.604990141982020  -0.275629004186377  -0.574329377582556  -0.077065712085292
		 0   0.388922234131299  -0.759948540113867   0.024826918339304  -0.520193556575719
-0.471404520791032   0.172854326280577  -0.389818163063590   0.296267892182356   0.712857836788948
	
	
R =
	
 4.242640687119286   7.306770072260991  -6.363961030678928   0.235702260395516  -5.421151989096865
-0.000000000000000   2.571208103423586  -0.972305585328246   3.219411826975751   1.015519166898391
-0.000000000000000   0.000000000000000   2.134156003843088  -4.858945588085554  -3.051606832063456
 0.000000000000000   0.000000000000000  -0.000000000000000   2.229457266869460   1.426720240565311
-0.000000000000000  -0.000000000000000   0.000000000000000   0.000000000000000  -0.481660700533072
	
	
A =
	
 1     3    -2     0    -2
 3     4    -5     1    -3
-2    -5     3    -2     2
 0     1    -2     5     3
-2    -3     2     3     4
	
	
Q*R =
	
 1.000000000000000   3.000000000000001  -2.000000000000000   0.000000000000000  -2.000000000000000
 3.000000000000000   4.000000000000000  -5.000000000000002   1.000000000000000  -3.000000000000000
-2.000000000000000  -5.000000000000000   3.000000000000000  -2.000000000000000   2.000000000000000
-0.000000000000000   1.000000000000000  -2.000000000000000   5.000000000000001   3.000000000000000
-2.000000000000000  -3.000000000000000   2.000000000000000   3.000000000000000   4.000000000000001
	\end{Verbatim}
	\fontsize{14pt}{14pt}\selectfont

	\subsubsection*{Вывод}
	\addcontentsline{toc}{subsubsection}{Вывод}
	Используя метод QR-разложения мы получаем две матрицы за n - итераций, где n это кол-во столбцов исходной матрицы А.
	Заметим, что при расчете матриц Q и R на месте нулей стоят очень маленькие значения, в том числе и отрицательные, это обусловлено вычислительными ошибками, которыми можно пренебречь.
	Перемножение найденных матриц Q и R дает нам исходную матрицу А с незначительной погрешностью.
	

	\pagebreak
	%%%%%%%%%%%%%%%%%%%%%%%%%%%%%%%%%%%%%%%%%%%%%%
	%%%%%%%%%%%%%%%%%%%%PAGE%?%%%%%%%%%%%%%%%%%%%%
	%%%%%%%%%%%%%%%%%%Заключение%%%%%%%%%%%%%%%%%%
	
	\section*{Заключение}
	\addcontentsline{toc}{section}{Заключение}
	
	В данной работе мы рассмотрели алгоритм построения QR-разложения методом отражений. Были проведены вычислительные эксперименты, демонстрирующие эффективность и точность данного метода.
	В результате работы над курсовым проектом были приобретены практические навыки владения:
	\begin{itemize}
		\item современному численному методу разложения матрицы;
		\item основами алгоритмизации для численного решения задач математической экономики на одном из языков программирования;
		\item инструментальными средствами, поддерживающими разработку программного обеспечения для численного решения задач математической экономики; 
	\end{itemize}
	а также навыками представления итогов проделанной работы в виде отчета, оформленного в соответствии с имеющимися требованиями, с привлечением современных средств редактирования и печати, а именно программы LaTeX.
	
	\pagebreak

	%%%%%%%%%%%%%%%%%%%%PAGE%?%%%%%%%%%%%%%%%%%%%%
	%%%%%%%%%%%%%%%%%%%Источники%%%%%%%%%%%%%%%%%%
	
	\section*{Список используемых источников}
	\addcontentsline{toc}{section}{Источники}
	
	\textbf{Источники}
	\begin{enumerate}
		\item Метод Хаусхолдера (отражений) QR-разложения квадратной матрицы, вещественный точечный вариант algowiki-project.org
		\item Численные методы. Андреев В.Б
		\item Вычисленные методы. Амосов А.А., Дубинский Ю.А. Копченова Н.В.
	\end{enumerate}	
	
\end{document}